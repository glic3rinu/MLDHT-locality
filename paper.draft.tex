ANNOUNCE CORRECTLY

Introduction
============
Distributed Hash Table (DHT) is a decentralized system that provides lookup services 
similar to hash tables; (key, values) pairs are stored in a DHT, and any particular 
node can efficiently retrieve the associated value. The descentralized nature and 
the scalability characteristics of DHT makes it the preferred way for storing shared 
state on P2P applications (e.g. peer lists). Kademlia and Mainline DHT are the most 
widley deployed DHTs world wide, creating overlay networks of milions of nodes [1].

[1] http://www.cs.helsinki.fi/u/jakangas/MLDHT/

However, DHTs not only serve good for storing peer swarms on P2P applications. These overlay 
networks are used on a large variety of Internet services. For example, CDNS [1, 2], 
application-level multicast, instant messaging, domain name services, web caching 
or distributed file systems. #TODO references for each example.

[1] Akamai
[2] Digital island

On the other hand, DHT technologies are mainly evolving thowards the Internet. 
Even though emerging networks like Communit Networks [??] are based on the same 
fundamental technology than Internet, they also have their own differences and peculiarities.

Community Networks raise from the dramatic drop of wireless network equipement 
cost over the last decade which has enabled end-users to build their own network 
independently from traditional telecomunications providers. This has a dramatic 
impact on the network properties. Most of its traffic flows through short-distance, 
low-throughput, high-latency wireless links, build with cheap low-end hardware.

The topology characteristics of the network are also a bit different from the traditional Internet. 
On village-like environments, typical CN deployment may involve a single Access Point 
on the top-roof of the villange from which all the clients contect to. On cities, 
usually the topology is a mesh-like network. Therefore, in both cases the network 
tends to forms clusters of nodes with a very few hops between each other but interconnected 
by long-distance intervilage wireless links.

Locality-aware applications become much important in CN because the cost of leaving 
_your neighberhood cluster_ is particularly expensive since it may involve going 
through several kilometers of unreliable microwave links, with significative packet loss, 
jitter, high latency and low throughput.

In this paper, we study the locality characteristics of DHT overlay on the context of Community 
Networks. In the next section background on DHT locality is presented.


Background
==========
Network proximity (locality) in Distributed Hash Tables has been extensiviely studied [1, 2, 3]. 
Three basic approaches have been suggested for exploring proximity in DHT protocols:

i) Geographic Layout
The node IDs are assigned in a manner that ensures that nodes that are close in the 
network topology are close in the node ID space. Geographic Layout was explored as 
one technique to improve routing performance in CAN[0]. In one implementation, 
nodes measure the RTT between themselves and a set of landmark servers to compute 
the coordinates opf the node in the CAN space.

ii) Proximity Routing
The routing tables are built without taking network proximity into account but the 
routing algorithm chooses a nearby node at each hop from among the ones in the 
routing table. Proximity routing was first proposed in CAN [0]. It involves no 
changes to routing table construction and maintenance, because routing tables are 
built without taking network proximity into account. TODO complete

iii) Proximity Neighbour Selection
Routing table construction takes network proximity into account. Routing table 
entries are chosen to refer to nodes that are nearby in the network topology, 
among all live nodes with appropiate node IDs. Proximity neighbour selection is 
used in Tapestry and Pastry.

[0] Exploring Network Proximity
[1] LDHT

Geographic Layout techniques are very effective for building overlays with a 
high degree of physical locality. However, we found that in the context of 
community networks it may not be the best choice. Network splits are common due to both, 
hardware malfunction (low-end hardware used) and humman errors during management 
operations (difficulty of operating a highly descentralized network). Because of network 
splits, a perfect locality based on Geographic Layout may lead to entire portions of the DHT ID space inaccessible. 
In this sense, Proximity Routing and Proximity Neighbour Selection provides much better 
reliability since the Node IDs have nothing to do with the geographic location.

Intuitively Proximity Routing may not work as good as Proximity Neighbour Selection. 
Because the routing table is small (compared to the total number of nodes), there 
are not a lot of candidates to choose from. Also priorizing pysical locality over 
logical locality may actually lead to an increase on the number of DHT hops for 
lookup operations.

We believe that Proximity Neighbour Selection offers the best characteristics for 
deployment on CN. ID space is uniformly distributed over the whole network, making 
it robust, while good enough locality can still be achieved by choosing good candidates 
as routing table entries.


Mainline DHT
============
We based our DHT evaluation on Mainline DHT (MLDHT). MLDHT is a Kademlia-based 
Distributed Hash Table used by BitTorrent clients to find peers via the BitTorrent 
protocol. MLDHT is the largest DHT network in the Internet, meassures from 2013 
show from 10 million to 25 million users, with a daily churn of at least 10 million[1].

Mainline DHT protocol is actually pretty simple, limited to four control messages:

1) PING: probe a node's availability. If the node fails to respond for some time, 
    it will be purged out of the routing table.
2) FIND_NODE: given a target ID, this message is used to find the {k} closest neighbors of the ID.
3) GET_PEERS: fiven an infohash, get the initial peer set. Notice that our evaluation 
    is not about the bittorrent peers returned by this query.
4) ANNOUNCE_PEER: a peer announces it belongs to a swarm.


Its simplicity and the fact that it is a widely deployed protocol makes its evaluation 
study somehow eassier, since the compleixity is low and there are several MLDHT 
implementations to choose from. We looked at libtorrent and PYMDHT.

libtorrent[1] was our first consideration, is a well known open source implementation 
of BitTorrent protocol and it has support for Mainline DHT. However, we quickly found out 
that its public API provides little information about the DHT routing state, only 
providing the node ID of its entrie, not IP addresses or RTT. This is not surprising since its designe is 
focused on BitTorrent aplications, not on DHT routing characteristics evaluation. 
Also libtorrent implements a highly tunne and complicated routing algorithm with a 
lot of corner cases and optimizations making its behaviour less predictable[2].

    [1] http://www.rasterbar.com/products/libtorrent/
    [2] http://cs.helsinki.fi/liang.wang/publications/P2P2013_13.pdf

PYMDHT [1] is a flexible Python implementation of the Mainline DHT protocol, 
specifically designed for MDHT evaluation. This fact has made us choose this 
library over libtorrent. Even though PYMDHT current state is experimental, 
the codebase has room for improvement and there is no public documentation available.

    [1] https://github.com/rauljim/pymdht

Mainline DHT as described in Bittorrent Enhacement Proposal 5 (BEP5) [1] does not 
contemplate means for locality. However, implementers of this DHT are free to choose 
whatever routing policy they wish to. In the case of PYMDHT it supports a plugin system 
for the routing policy. The library ships with an implementation of a Proximity Neighbour 
Selection routing policy defined by NICE[2], based on RTT meassurements, which 
is very interesting for our locality evaluation.

    [1] http://www.bittorrent.org/beps/bep_0005.html
    [2] ???



MLDHT Locality Evaluation on Community Lab
==========================================
The evaluation has been performed in Community Lab [1], a Community Networks Testbed 
by the CONFINE project[2], is a global facility for experimentation with network 
technologies and services for community networks.

Interestingly, Community Lab topology has also the type of node clustering commonly 
seen in community network villages. In the case of Community Lab those are the nodes 
at UPC-Lab where only one forwarding device sits between every node[1].

[1] http://monitor.confine-project.eu:8000/networktrace/




* LARGE SET OF CANDIDATE NODES


* characteristics of the MDHT
* how we can improve locality theoretically
* Why Proximity Neighbour Selection is the best approach and the others are crap
* Considterations: Small set of candidates decreases performance, size of the dht

* Even that wireless links have a lot of jitter (latency variation) we decided to 
use the RTT metric because it was the most feasible to use.

* Two routing policies: BEP5 vs NICE RTT (Proximity Neighbour Selection)
* Tune DHT size
* Write a simple client that:
    * Bootstrap a PYMDHT
    * Fill ID's 
    * make queries
* Routing table snapshoting
* Compare results
* CHURN


Because of the reduced number of available nodes in our testbed (in relation to 
the massive DHT size in real deployments) we have found that we are not able to 
reproduce meaningful results of the locality characteristics of MLDHT. 
Therefore, we have crafted some workloads trying to artificaly promote locality 
rather than focus on real workload behaviours.

NUM_NODES_PER_BUCKET = 1

Not able to reduce the ID space for library limitations

High load workload
------------------
Each DHT node publishes N info hashes and perform get_peers() of all the infohashes 
that the other nodes have published (NUM_NODES-1)*NUM_HASHES of total queries every minute.

We made sure that the cache size of each node is large enough to hold all the IDs. 
Because of caching *ONE MAY EXPECT THAT* we expected data to eventually propagate 
to all nodes. Therefore, at some point the locality of the routing table should start 
to stabilize and converge to entries with low RTT (and thus more local).

*All routes

cache size: 10000 entries vs 100 entries vs cache off


Crawded bucket workload
-----------------------
All our nodes have an ID within the same bucket, and all nodes perform get_peers 
queries of the same infohash that is also on the same bucket.

We expect routing table entries RTT to converge to an small value

Lonely node workload
--------------------

This is a variation of the crowded bucket workload. in this case all nodes are 
members of the same bucket except one, that its ID belongs to the fardest bucket.

The lonley node is selected from a cluster of nodes (lab) because it has a lot of neighbors 

We expected the lonely node to select a very good entry for the crowded bucket. 
But surprisingly the routing table remained empty, despite of the lonley node being 
able to send and recieve get_peers queries and responses.


Future Work
===========
* Further analysis should be performed in order to better define the locality characteristics of MLDHT.
* how does a locality-aware routing policy affects response time latency and other MDHT performance characteristics.
* meassure locality not only on the routing table (first hop), but also on the total number of hops a request goes through.
* Test with different workloads to simulate different realitstic scenarios. Different degree of simultaneous requests
* Bigger DHT size in order to perform realistic workload experiments. We haved to carefully design the scencario to test locality but if the number of nodes was bigger we ...
* Current state of tools for MDHT evaluation is not mature enough for our background and scheduled time for our project.

* Communitylab suffers from interconnection problems between community networks, also Internet connectivity is not widely available; considerably reducing the number of useful nodes. Because, nodes don't always see each other and since Internet is not always available it increases the complexity of deploying our experiment in those disconnected nodes beyond what we consider feasible for our project's schedule.
* Reasons for low number of nodes: offline, no Internet and isolated, lack of CN public ips


Conclusions
===========
We did not proved it but improving locality on MDHT sounds theoretically plausible

Locality in MDHT seems non-existent under the analyzed workloads. Even when routing policies take into account physical information like Round trip time.

Naive changes on the routing table policy in order to improve locality may actually 
harm locality because of the nature of the MLDHT. More hops? disrupt logical locality?

References
==========
http://dl.acm.org/citation.cfm?id=1146908
